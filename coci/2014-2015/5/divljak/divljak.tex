\renewcommand{\taskname}{DIVLJAK}
\renewcommand{\timelimit}{4 sekunde}
\renewcommand{\memorylimit}{768 MB}
\renewcommand{\score}{160 bodova}

U današnje vrijeme puno je neobičnih osoba, no nećemo ulaziti u detalje već ćemo se samo posvetiti određenom tipu, nama osobno najzanimljivijim osobama. Naravno, riječ je o divljacima.

Puno je divljaka, no malo je onih stvarno bitnih. U ovoj priči nalazi se $N$ bitnih divljaka, označenih brojevima od $1$ do $N$. Svaki od njih ima svoju kamenu ploču na kojoj piše njegova riječ, koja se sastoji isključivo od malih slova engleske abecede. 

Naši divljaci igraju jednu zanimljivu igru sa svojim dobrim prijateljem Tarzanom.

Igra se odvija u $Q$ rundi. Postoje dva oblika runde, a oblik svake runde određuje Tarzan:
{
\setitemize[0]{leftmargin=60pt}
\begin{itemize}
\item[1. oblik:] Tarzan pokaže divljacima riječ $P$.
\item[2. oblik:] Tarzan upita divljaka s oznakom $S$ sljedeće pitanje: “Od svih riječi koje sam do sada pokazao, koliko je onih kojima je riječ na tvojoj kamenoj ploči bila uzastopni podniz?”
\end{itemize}}

S obzirom da divljaci puno divljaju i shodno tome nisu u stanju cijelo vrijeme pratiti što se događa tijekom igre, oni trebaju vašu pomoć. Pomozite divljacima da točno odgovore na svako pitanje koje im Tarzan postavi.

\strut

\naslov{ulazni podaci}

U prvom retku nalazi se jedan prirodni broj N ($1 \leqslant N \leqslant 10^5$), broj divljaka.

U idućih $N$ redaka nalazi se po jedna riječ koja se sastoji isključivo od malih slova engleske abecede, $i$-ta od tih riječi odgovara riječi na kamenoj ploči divljaka s oznakom $i$.

Nakon toga slijedi prirodan broj $Q$ ($1 \leqslant Q \leqslant 10^5$), broj rundi igre.

Idućih $Q$ redaka opisuju runde igre, $i$-ti od tih redaka opisuje $i$-tu rundu igre.
U svakom retku nalazit će se broj $O$. U slučaju da je $O$ jednak 1 radi se o prvom tipu runde i u istom retku slijedi pokazana riječ $P$ koja se sastoji samo od malih slova engleske abecede.

U slučaju da je $O$ jednak 2 radi se o drugom tipu runde i u istom retku slijedi broj $S$ ($1 \leqslant S \leqslant N$), oznaka divljaka kojemu je Tarzan postavio pitanje.

Ukupna duljina svih riječi koje pišu na pločama divljaka neće biti veća od $2\cdot10^6$.
Ukupna duljina svih riječi koje Tarzan pokazuje divljacima neće biti veća od $2\cdot10^6$.

\strut

\naslov{izlazni podaci}

Za svaku rundu drugog oblika ispišite po jedan redak. U $i$-tom ispisanom retku mora se nalaziti točan odgovor na Tarzanovo pitanje u $i$-toj rundi drugog oblika.

\strut

\naslov{bodovanje}

U test podacima ukupno vrijednima 50\% bodova vrijedit će $N \leqslant 20\,000$.

\pagebreak

\naslov{primjeri test podataka}

\begin{center}
\fontfamily{\ttdefault}
\fontsize{10pt}{1em}
\selectfont
\begin{tabu}to 0.99\textwidth{|X[1]|X[1]|}
\hline
& \\ 
\rowfont{\fontsize{10pt}{1em}\bfseries}
ulaz & ulaz \\
\verbatiminput{divljak/divljak.dummy.in.1} & 
\verbatiminput{divljak/divljak.dummy.in.2} \\
\rowfont{\fontsize{10pt}{1em}\bfseries}
izlaz & izlaz \\
\verbatiminput{divljak/divljak.dummy.out.1} & 
\verbatiminput{divljak/divljak.dummy.out.2} \\
\hline
\end{tabu}
\end{center}

{
\fontsize{10pt}{1em}
\selectfont
\textbf{Pojašnjenje prvog primjera:} Jedina riječ koju je Tarzan izrekao je abca. Odgovor na prvi upit je naravno 1 jer se riječ a nalazi u riječi abca. Odgovor na drugi upit je također 1 jer se riječ abc nalazi u riječi abca. \\
%\textbf{Pojašnjenje drugog primjera:} pojašnjenje pojašnjenje pojašnjenje
}