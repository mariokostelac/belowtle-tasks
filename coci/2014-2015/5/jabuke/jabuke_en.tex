\renewcommand{\taskname}{JABUKE}
\renewcommand{\timelimit}{2 seconds}
\renewcommand{\memorylimit}{128 MB}
\renewcommand{\score}{140 points}

It is often heard that the apple doesn't fall far from the tree. But is that really so?
%Često čujemo ljude kako govore da jabuka ne pada daleko od stabla. No je li to stvarno istina? 

The National Statistics Department has tracked the falling of apples in an orchard for $G$ consecutive years. The orchard can be represented as a matrix with dimensions $R \cdot S$. Each field of the matrix can contain more than one apple tree.
%Državni zavod za statistiku (DZS) $G$ uzastopnih godina pratio je padanje jabuka u jednom voćnjaku. Taj voćnjak možemo prikazati kao $R \cdot S$ matricu. U svakom polju matrice može se nalaziti više stabala jabuka.

Interestingly enough, each year there was exactly one apple fall, so the Department decided to write down $G$ pairs of numbers $(r_i, s_i)$ that denote the row and column of the location where the apple fell during the $i$\textsuperscript{th} year. Moreover, \textbf{by next year}, a new tree grew at that location.
%Zanimljivo, svake godine pala je točno jedna jabuka pa su u zavodu zapisali $G$ parova brojeva $(r_i, s_i)$ koji označavaju redak i stupac mjesta na koje je pala jabuka $i$-te godine. Također, na tom je mjestu \textbf{do iduće godine} niknulo novo stablo.

Your task is to determine the squared distance between the nearest tree and the apple that fell, measured in unit fields of the matrix (we assume it is that tree from which the apple fell).
%Vaš je zadatak za svaku jabuku koja je pala odrediti kvadrat udaljenosti do najbližeg stabla jabuke mjerenu u jediničnim poljima matrice (pretpostavljamo da je to stablo s kojeg je jabuka pala).

The distance between fields $(r_1, s_1)$ and $(r_2, s_2)$ in the matrix are calculated as:
%Udaljenost između polja $(r_1, s_1)$ i $(r_2, s_2)$ u matrici računamo kao:
\[ d((r_1, s_1), (r_2, s_2)) = \sqrt{(r_1-r_2)^2 + (s_1-s_2)^2} \]

\strut

\naslov{input}

The first line of input contains two integers, $R$ and $S$ ($1 \leqslant R, S \leqslant 500$), the number of rows and columns of the matrix.
%U prvom retku nalaze se dva prirodna broja, $R$ i $S$ ($1 \leqslant R, S \leqslant 500$), broj redaka i stupaca matrice.

Each of the following $R$ lines contains $S$ characters 'x' or '.'. The character '.' denotes an empty field, and the character 'x' denotes a field with at least one tree.
%U idućih $R$ redaka nalazi se $S$ znakova 'x' ili '.'.  Znak '.' označava prazno polje, a znak 'x' polje na kojem se nalazi jedno stablo.

The orchard will initially contain at least one tree.
%U voćnjaku će se na početku nalaziti barem jedno stablo.

After that, an integer $G$ ($1 \leqslant G \leqslant 10^5$) follows, the number of years the orchard has been under observation.
%Nakon toga slijedi prirodan broj $G$ ($1 \leqslant G \leqslant 10^5$), broj godina u kojima je voćnjak promatran.

Each of the following $G$ lines describes the falls of the apples. Each line contains a pair of integers $(r_i, s_i)$ that denote the row and column of the location where the apple fell in the $i$\textsuperscript{th} year.
%Idućih $G$ redova opisuje padove jabuka. U svakom retku nalazi se jedan par prirodnih brojeva $(r_i, s_i)$ koji predstavljaju redak i stupac u koji je pala jabuka $i$-te godine. 

\strut

\naslov{output}

Output $G$ numbers, the required squared distances from the task, each in its own line.
%Ispišite $G$ brojeva, tražene kvadrate udaljenosti iz teksta zadatka, svaki u svom retku.

\strut

\naslov{scoring}

In test cases worth 30\% of total points, it will hold $G \leqslant 500$.
%U test podacima ukupno vrijednima 30\% bodova vrijedit će $G \leqslant 500$.

\pagebreak

\naslov{sample tests}

\begin{center}
\fontfamily{\ttdefault}
\fontsize{10pt}{1em}
\selectfont
\begin{tabu}to 0.99\textwidth{|X[1]|X[1]|X[1]|}
\hline
& & \\ 
\rowfont{\fontsize{10pt}{1em}\bfseries}
input & input & input\\
\verbatiminput{jabuke/jabuke.dummy.in.1} &
\verbatiminput{jabuke/jabuke.dummy.in.1} & 
\verbatiminput{jabuke/jabuke.dummy.in.2} \\
\rowfont{\fontsize{10pt}{1em}\bfseries}
output & output & output\\
\verbatiminput{jabuke/jabuke.dummy.out.1} &
\verbatiminput{jabuke/jabuke.dummy.out.1} & 
\verbatiminput{jabuke/jabuke.dummy.out.2} \\
\hline
\end{tabu}
\end{center}

{
\fontsize{10pt}{1em}
\selectfont
\textbf{Clarification of the first example:} The closest apple to the one that fell in the first year is the apple in the field (1,1). The apple that fell in the second year fell on the exact field where the tree is located, so the squared distance is 0. The apple that fell in the third year is equally distant to all three existing trees in the orchard.
%{Pojašnjenje prvog primjera:} Najbliža jabuka onoj koja je pala prve godine jest jabuka u polju (1,1). Jabuka koja je pala druge godine je pala u samo polje gdje se nalazi stablo pa je kvadrat udaljenosti jednak 0. Jabuka koja je pala treće godine jednako je udaljena od sva tri postojeća stabla u voćnjaku. \\
%\textbf{Pojašnjenje drugog primjera:} pojašnjenje pojašnjenje pojašnjenje
}