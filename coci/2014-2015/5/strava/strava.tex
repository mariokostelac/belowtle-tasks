\renewcommand{\taskname}{STRAVA}
\renewcommand{\timelimit}{1 sekunda}
\renewcommand{\memorylimit}{32 MB}
\renewcommand{\score}{20 bodova}

Duž Ulice brijestova \textbf{s obje strane} posađeni su, pogađate, brijestovi. Svaka dva susjedna brijesta međusobno su razmaknuta $K$ metara te je poznato da se na početku i kraju ulice nalazi po jedan brijest. Nakon nemilih događaja, stanovnici ulice odlučili su svaki treći brijest zamijeniti hrastom. 

Ako znate da je ulica dugačka $L$ metara, odredite koliko je brijestova, a koliko hrastova posađeno u Ulici brijestova. 

\strut

\naslov{ulazni podaci}

U jedinom redu ulaza nalaze se brojevi $K$ i $L$ ($1 \leqslant K < L \leqslant 1000$, $L$ je djeljiv s $K$) iz teksta zadatka.

\strut

\naslov{izlazni podaci}

Ispišite dva cijela broja odvojena razmakom koji redom predstavljaju broj brijestova i broj hrastova koji se nalaze u Ulici brijestova.

\textbf{Napomena:} brojevi $K$ i $L$ bit će takvi da se u ulici nalaze barem 3 stabla.

\strut

\naslov{primjeri test podataka}

\begin{center}
\fontfamily{\ttdefault}
\fontsize{10pt}{1em}
\selectfont
\begin{tabu}to 0.99\textwidth{|X[1]|X[1]|X[1]|}
\hline
& & \\ 
\rowfont{\fontsize{10pt}{1em}\bfseries}
ulaz & ulaz & ulaz\\
\verbatiminput{strava/strava.dummy.in.1} &
\verbatiminput{strava/strava.dummy.in.2} & 
\verbatiminput{strava/strava.dummy.in.3} \\
\rowfont{\fontsize{10pt}{1em}\bfseries}
izlaz & izlaz & izlaz\\
\verbatiminput{strava/strava.dummy.out.1} &
\verbatiminput{strava/strava.dummy.out.2} & 
\verbatiminput{strava/strava.dummy.out.3} \\
\hline
\end{tabu}
\end{center}

{
\fontsize{10pt}{1em}
\selectfont
%\textbf{Pojašnjenje prvog primjera:} pojašnjenje pojašnjenje pojašnjenje \\
%\textbf{Pojašnjenje drugog primjera:} pojašnjenje pojašnjenje pojašnjenje
}