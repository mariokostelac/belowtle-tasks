\renewcommand{\taskname}{TRAKTOR}
\renewcommand{\timelimit}{2 sekunde}
\renewcommand{\memorylimit}{32 MB}
\renewcommand{\score}{100 bodova}

Mirko je za Božić dobio novi super traktor kojim može brati gljive! Gljive rastu na livadi kvadratnog oblika koju možemo smjestiti u koordinatnu ravninu tako da joj je donji lijevi rub u točki $(1, 1)$ a gornji desni u točki $(10^5, 10^5)$.

U početnom trenutku na livadi nema gljiva, no narast će ih ukupno $N$ i to tako da svake sekunde naraste točno jedna nova na nekom praznom mjestu na livadi. 

Štedljivi Mirko želi \emph{samo jednom} vožnjom traktora ubrati barem $K$ gljiva. Svoj put započinje u jednoj od točaka livade i može voziti samo paralelno njenim stranicama ili dijagonalama. 
Mirkov traktor je super brz, \textbf{proizvoljno veliku udaljenost prelazi u zanemarivom vremenu}. Zbog silne brzine, Mirko \emph{ne može skretati} za vrijeme vožnje.

Pomozite Mirku i odredite \textbf{najmanji broj sekundi} nakon kojih može ubrati željeni broj gljiva. 

\strut

\naslov{ulazni podaci}

U prvom retku nalaze se prirodni brojevi $N$ ($2 \leqslant N \leqslant 10^6$) i $K$ ($2 \leqslant K \leqslant N$), broj gljiva koje će izrasti i broj gljiva koji Mirko želi ubrati.

U sljedećih $N$ redaka nalaze se po dva prirodna broja $X_i$ i $Y_i$ ($1 \leqslant X_i, Y_i \leqslant 10^5$), koordinate $i$-te gljive koja je narasla na livadi.

\naslov{izlazni podaci}

U jedini redak ispišite traženi najmanji broj sekundi.
Ako Mirko ne može ubrati $K$ gljiva jednom vožnjom, ispišite -1.

\naslov{bodovanje}

U test podacima vrijednima ukupno 50\% bodova vrijedit će $1 \leqslant X_i, Y_i \leqslant 300$.

\strut

\naslov{primjeri test podataka}

\begin{center}
\fontfamily{\ttdefault}
\fontsize{10pt}{1em}
\selectfont
\begin{tabu}to 0.99\textwidth{|X[1]|X[1]|X[1]|}
\hline
& & \\ 
\rowfont{\fontsize{10pt}{1em}\bfseries}
ulaz & ulaz & ulaz\\
\verbatiminput{traktor/traktor.dummy.in.1} &
\verbatiminput{traktor/traktor.dummy.in.2} & 
\verbatiminput{traktor/traktor.dummy.in.3} \\
\rowfont{\fontsize{10pt}{1em}\bfseries}
izlaz & izlaz & izlaz\\
\verbatiminput{traktor/traktor.dummy.out.1} &
\verbatiminput{traktor/traktor.dummy.out.2} & 
\verbatiminput{traktor/traktor.dummy.out.3} \\
\hline
\end{tabu}
\end{center}

{
\fontsize{10pt}{1em}
\selectfont
\textbf{Pojašnjenje prvog primjera:} Mirko svoju vožnju započinje u točki $(1, 2)$ i kreće se prema gljivi s koordinatama $(4, 5)$. \\
%\textbf{Pojašnjenje drugog primjera:} pojašnjenje pojašnjenje pojašnjenje
}