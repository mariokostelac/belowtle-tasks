\renewcommand{\taskname}{ZGODAN}
\renewcommand{\timelimit}{1 second}
\renewcommand{\memorylimit}{32 MB}
\renewcommand{\score}{120 points}

An integer is considered handsome if it has every two consecutive digits of different parity.
%Prirodan broj nazivamo zgodnim ako su mu svake dvije susjedne znamenke različite parnosti.
For a given integer $N$, what is its closest handsome number?
%Za dani prirodan broj $N$, koji je njemu najbliži zgodni broj?

\textbf{Please note:} Numbers consisting of only one digit are handsome numbers. The distance of two numbers is the absolute value of their difference.
%\textbf{Napomene:} Jednoznamenkasti brojevi su zgodni. Udaljenost dvaju brojeva apsolutna je vrijednost njihove razlike.

\strut

\naslov{input}

The first and only line of input contains the integer $N$ that consists of a thousand digits at most and is not handsome.
%U jedinome retku nalazi se prirodan broj $N$ koji ima najviše tisuću znamenaka i nije zgodan.

\strut
\naslov{output}

The first and only line of output must contain the required closest handsome number. If two closest numbers exist, output both, separated by space.
%U jedini redak ispišite traženi najbliži zgodni broj. Ako postoje dva najbliža broja, ispišite oba, odvojene razmakom.

\strut
\naslov{scoring}

In test cases worth 56 points, it will hold $N < 10^9$.
%U test podacima ukupno vrijednima 56 bodova vrijedit će $N < 10^9$.

\strut

\naslov{sample tests}

\begin{center}
\fontfamily{\ttdefault}
\fontsize{10pt}{1em}
\selectfont
\begin{tabu}to 0.99\textwidth{|X[1]|X[1]|}
\hline
& \\ 
\rowfont{\fontsize{10pt}{1em}\bfseries}
input & input \\
\verbatiminput{zgodan/zgodan.dummy.in.1} &
\verbatiminput{zgodan/zgodan.dummy.in.2} \\
\rowfont{\fontsize{10pt}{1em}\bfseries}
output & output \\
\verbatiminput{zgodan/zgodan.dummy.out.1} &
\verbatiminput{zgodan/zgodan.dummy.out.2} \\
\hline
\end{tabu}
\end{center}

{
\fontsize{10pt}{1em}
\selectfont
%\textbf{Pojašnjenje prvog primjera:} pojašnjenje pojašnjenje pojašnjenje \\
%\textbf{Pojašnjenje drugog primjera:} pojašnjenje pojašnjenje pojašnjenje
}