\renewcommand{\taskname}{ZMIJA}
\renewcommand{\timelimit}{1 sekunda}
\renewcommand{\memorylimit}{32 MB}
\renewcommand{\score}{80 bodova}

Mirko radi kopiju popularne računalne igre "Zmijica". U igri kontrolirate zmijičine kretnje po ekranu dimenzija $R \cdot S$ piksela, a cilj igre je pokupiti sve jabučice koje se nalaze na ekranu. 

Nažalost, Mirkova implementacija nije baš bajna pa se tijek igre razlikuje od originala. Slijedi opis Mirkove igre:
\begin{itemize}
\item za razliku od originala, jabučice se ne pojavljuju nasumično na ekranu, već su na početku igre poznate pozicije svih jabuka
\item na početku igre, zmijica se nalazi u donjem lijevom kutu ekrana i gleda udesno
\item postoje dvije tipke u igri, označene s A i B
\item pritiskom na tipku A, zmijica se pomiče za jedan piksel u smjeru u kojem trenutno gleda. Ako bi zmijica tim pomakom izašla iz ekrana, onda se ne događa ništa.
\item pritiskom na tipku B, zmijica se pomiče za jedan piksel prema gore i mijenja smjer gledanja za 180\textdegree
\item kada se zmijica pomakne na piksel na kojem se nalazi jabuka, pojede ju, ali ne naraste kao u originalu
\end{itemize}

Vaš zadatak je sljedeći: za zadane pozicije jabučica na početku igre odredite \textbf{najmanji broj pritisaka tipki} potreban da zmijica skupi \textbf{sve jabučice}.

\strut

\naslov{ulazni podaci}

U prvom retku nalaze se prirodni brojevi $R$ i $S$ ($2 \leqslant R, S \leqslant 1\,000$), visina i širina ekrana.

U svakom od idućih $R$ redaka nalazi se točno $S$ znakova. Ti znakovi predstavljaju sadržaj ekrana. Na pikselima gdje su jabučice bit će znak 'J', na praznim pikselima znak '.'.

U donjem lijevom kutu nalazit će se znak 'Z' koji predstavlja zmijicu na njenoj početnoj poziciji.

\strut

\naslov{izlazni podaci}

U jedini redak izlaza ispišite traženi najmanji broj pritisaka tipki.

\strut

\naslov{primjeri test podataka}

\begin{center}
\fontfamily{\ttdefault}
\fontsize{10pt}{1em}
\selectfont
\begin{tabu}to 0.99\textwidth{|X[1]|X[1]|X[1]|}
\hline
& & \\ 
\rowfont{\fontsize{10pt}{1em}\bfseries}
ulaz & ulaz & ulaz\\
\verbatiminput{zmija/zmija.dummy.in.1} &
\verbatiminput{zmija/zmija.dummy.in.2} & 
\verbatiminput{zmija/zmija.dummy.in.3} \\
\rowfont{\fontsize{10pt}{1em}\bfseries}
izlaz & izlaz & izlaz\\
\verbatiminput{zmija/zmija.dummy.out.1} &
\verbatiminput{zmija/zmija.dummy.out.2} & 
\verbatiminput{zmija/zmija.dummy.out.3} \\
\hline
\end{tabu}
\end{center}

{
\fontsize{10pt}{1em}
\selectfont
\textbf{Pojašnjenje prvog primjera:} Najkraći niz pritisaka tipki potreban da zmijica skupi sve jabučice je BBAAABB. \\
%\textbf{Pojašnjenje drugog primjera:} pojašnjenje pojašnjenje pojašnjenje
}