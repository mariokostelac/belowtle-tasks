\documentclass[11pt, oneside, a4paper]{article}

\usepackage{courier}

\usepackage[
	includehead,
	includefoot,
	headheight=30pt,
	left=0.39in,
	right=0.39in,
	top=0.39in,
	bottom=0.79in,
	headsep=10pt,
	landscape
]{geometry}

\usepackage{fancyhdr}
\usepackage{graphicx}
\usepackage{verbatim}
\usepackage{parskip}

\usepackage{tabu}
\usepackage{xcolor}
\usepackage{colortbl}

\renewcommand{\ttdefault}{pcr}
\usepackage[utf8]{inputenc}
\usepackage[T1]{fontenc}

\lhead{%
	\fontfamily{\sfdefault}
	\fontsize{10pt}{1.1em}
	\bfseries
	\selectfont
	HONI 2014/2015%%
}	

\rhead{%
	\fontfamily{\sfdefault}
	\fontsize{10pt}{1.1em}
	\bfseries
	\selectfont
	5. kolo, 18. siječnja 2015.%
}

\cfoot{}

\renewcommand{\headrule}{
\kern 1pt
\hrule width \hsize
\kern 1pt
\hrule width \hsize
}

\newcommand{\naslov}[1]{
\begin{center}
\fontfamily{\sfdefault}
\fontsize{10pt}{1.1em}
\bfseries
\selectfont
\uppercase{#1}
\end{center}
}

\setlength{\parindent}{0pt}

\pagestyle{fancy}

\begin{document}

\fontfamily{\sfdefault}
\fontsize{10pt}{1.1em}
\selectfont

\rule[-8\baselineskip]{0pt}{\baselineskip}

\extrarowsep = 7pt

\colorlet{headcolor}{gray!25}

\begin{tabu}to\linewidth{|[1.5pt]c|[1.5pt]X[-1c]|X[-0.9c]|X[-1c]|X[-0.95c]|X[-1c]|X[-1c]|X[-1c]|X[-1c]|[1.5pt]}
\tabucline[1.5pt]-
\rowfont{\bfseries}
\rowcolor{headcolor}
ZADATAK & STRAVA & UŽAS & FUNGHI & ZMIJA & TRAKTOR & ZGODAN & JABUKE & DIVLJAK \\
\tabucline[1.5pt]-
\textbf{izvorni k\^{o}d} & 
\extrarowsep = 0pt
\begin{tabular}{c}
\texttt{strava.pas} \\
\texttt{strava.c} \\
\texttt{strava.cpp} \\
\texttt{strava.py} \\
\texttt{strava.java} \\
\end{tabular}
&
\extrarowsep = 0pt
\begin{tabular}{c}
\texttt{uzas.pas} \\
\texttt{uzas.c} \\
\texttt{uzas.cpp} \\
\texttt{uzas.py} \\
\texttt{uzas.java} \\
\end{tabular}
& 
\extrarowsep = 0pt
\begin{tabular}{c}
\texttt{funghi.pas} \\
\texttt{funghi.c} \\
\texttt{funghi.cpp} \\
\texttt{funghi.py} \\
\texttt{funghi.java} \\
\end{tabular}
& 
\extrarowsep = 0pt
\begin{tabular}{c}
\texttt{zmija.pas} \\
\texttt{zmija.c} \\
\texttt{zmija.cpp} \\
\texttt{zmija.py} \\
\texttt{zmija.java} \\
\end{tabular}
& 
\extrarowsep = 0pt
\begin{tabular}{c}
\texttt{traktor.pas} \\
\texttt{traktor.c} \\
\texttt{traktor.cpp} \\
\texttt{traktor.py} \\
\texttt{traktor.java} \\
\end{tabular}
& 
\extrarowsep = 0pt
\begin{tabular}{c}
\texttt{mravi.pas} \\
\texttt{mravi.c} \\
\texttt{mravi.cpp} \\
\texttt{mravi.py} \\
\texttt{mravi.java} \\
\end{tabular}
& 
\extrarowsep = 0pt
\begin{tabular}{c}
\texttt{jabuke.pas} \\
\texttt{jabuke.c} \\
\texttt{jabuke.cpp} \\
\texttt{jabuke.py} \\
\texttt{jabuke.java} \\
\end{tabular}
& 
\extrarowsep = 0pt
\begin{tabular}{c}
\texttt{divljak.pas} \\
\texttt{divljak.c} \\
\texttt{divljak.cpp} \\
\texttt{divljak.py} \\
\texttt{divljak.java} \\
\end{tabular}
\\
\tabucline-
\textbf{ulazni podaci} & \multicolumn{8}{c|[1.5pt]}{standardni ulaz} \\
\tabucline-
\textbf{izlazni podaci} & \multicolumn{8}{c|[1.5pt]}{standardni izlaz} \\
\tabucline-
\textbf{vremensko ograničenje} & 1 sekunda & 1 sekunda & 1 sekunda & 1 sekunda & 2 sekunde & 1 sekunda & 2 sekunde & 4 sekunde \\
\tabucline-
\textbf{memorijsko ograničenje} & 32 MB & 32 MB & 32 MB & 32 MB & 32 MB & 32 MB & 128 MB & 768 MB \\
\tabucline-
\rowfont{\bfseries}
 & 20 & 30 & 50 & 80 & 100 & 120 & 140 & 160 \\
\tabucline{2-9}
\textbf{broj bodova} & \multicolumn{8}{c|[1.5pt]}{\textbf{ukupno 700, maksimalno 600}} \\
& \multicolumn{8}{c|[1.5pt]}{(natjecatelju se zbrajaju bodovi onih 5 zadataka na kojima je ostvario najviše bodova)} \\
\tabucline[1.5pt]-
\end{tabu}


\end{document}