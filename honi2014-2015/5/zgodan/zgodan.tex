\renewcommand{\taskname}{ZGODAN}
\renewcommand{\timelimit}{1 sekunda}
\renewcommand{\memorylimit}{32 MB}
\renewcommand{\score}{120 bodova}

Prirodan broj nazivamo zgodnim ako su mu svake dvije susjedne znamenke različite parnosti.
Za dani prirodan broj $N$, koji je njemu najbliži zgodni broj?

\textbf{Napomene:} Jednoznamenkasti brojevi su zgodni. Udaljenost dvaju brojeva apsolutna je vrijednost njihove razlike.

\strut

\naslov{ulazni podaci}

U jedinome retku nalazi se prirodan broj $N$ koji ima najviše tisuću znamenaka i nije zgodan.

\strut
\naslov{izlazni podaci}

U jedini redak ispišite traženi najbliži zgodni broj. Ako postoje dva najbliža broja, ispišite oba, odvojene razmakom.

\strut
\naslov{bodovanje}

U test podacima ukupno vrijednima 56 bodova vrijedit će $N < 10^9$.

\strut

\naslov{primjeri test podataka}

\begin{center}
\fontfamily{\ttdefault}
\fontsize{10pt}{1em}
\selectfont
\begin{tabu}to 0.99\textwidth{|X[1]|X[1]|}
\hline
& \\ 
\rowfont{\fontsize{10pt}{1em}\bfseries}
ulaz & ulaz \\
\verbatiminput{zgodan/zgodan.dummy.in.1} &
\verbatiminput{zgodan/zgodan.dummy.in.2} \\
\rowfont{\fontsize{10pt}{1em}\bfseries}
izlaz & izlaz \\
\verbatiminput{zgodan/zgodan.dummy.out.1} &
\verbatiminput{zgodan/zgodan.dummy.out.2} \\
\hline
\end{tabu}
\end{center}

{
\fontsize{10pt}{1em}
\selectfont
%\textbf{Pojašnjenje prvog primjera:} pojašnjenje pojašnjenje pojašnjenje \\
%\textbf{Pojašnjenje drugog primjera:} pojašnjenje pojašnjenje pojašnjenje
}