\renewcommand{\taskname}{DIVLJAK}
\renewcommand{\timelimit}{4 seconds}
\renewcommand{\memorylimit}{768 MB}
\renewcommand{\score}{160 points}

Nowadays, there are a lot of unusual people. We won't go into details, but instead focus on a certain type, to us personally the most interesting people. Of course, we're talking about barbarians!
%U današnje vrijeme puno je neobičnih osoba, no nećemo inputiti u detalje već ćemo se samo posvetiti određenom tipu, nama osobno najzanimljivijim osobama. Naravno, riječ je o divljacima.

There are a lot of barbarians, but only a few of them are truly important. This story has $N$ important barbarians, denoted with integers from $1$ to $N$. Each of them has their own stone tablet with their word written on it, consisting of only lowercase letters of the English alphabet.
%Puno je divljaka, no malo je onih stvarno bitnih. U ovoj priči nalazi se $N$ bitnih divljaka, označenih brojevima od $1$ do $N$. Svaki od njih ima svoju kamenu ploču na kojoj piše njegova riječ, koja se sastoji isključivo od malih slova engleske abecede. 

Our barbarians are playing an interesting game with their good friend Tarzan.
%Naši divljaci igraju jednu zanimljivu igru sa svojim dobrim prijateljem Tarzanom.

The game is played in $Q$ rounds. There are two round types and each is determined by Tarzan:
%Igra se odvija u $Q$ rundi. Postoje dva oblika runde, a oblik svake runde određuje Tarzan:
{
\setitemize[0]{leftmargin=60pt}
\begin{itemize}
\item[1\textsuperscript{st} type:] Tarzan show the word $P$ to the barbarians.
\item[2\textsuperscript{nd} type:] Tarzan ask the barbarian denoted with $S$ the following question: “Out of all the words I've shown you so far, how many of them are such that the word on your stone tablet is their consecutive substring?”
%\item[1. oblik:] Tarzan pokaže divljacima riječ $P$.
%\item[2. oblik:] Tarzan upita divljaka s oznakom $S$ sljedeće pitanje: “Od svih riječi koje sam do sada pokazao, koliko je onih kojima je riječ na tvojoj kamenoj ploči bila uzastopni podniz?”
\end{itemize}}

Given the fact that the barbarians go wild a lot and aren't really able to pay attention and keep up what's happening in the game, they need your help. Help the barbarians answer each of Tarzan's questions correctly.
%S obzirom da divljaci puno divljaju i shodno tome nisu u stanju cijelo vrijeme pratiti što se događa tijekom igre, oni trebaju vašu pomoć. Pomozite divljacima da točno odgovore na svako pitanje koje im Tarzan postavi.

\strut

\naslov{input}

The first line of input contains the integer $N$ ($1 \leqslant N \leqslant 10^5$), the number of barbarians.
%U prvom retku nalazi se jedan prirodni broj N ($1 \leqslant N \leqslant 10^5$), broj divljaka.

Each of the following $N$ lines contains a single word consisting of only lowercase letters of the English alphabet, the $i$\textsuperscript{th} word corresponding to the word on the stone tablet of barbarian denoted with $i$.
%U idućih $N$ redaka nalazi se po jedna riječ koja se sastoji isključivo od malih slova engleske abecede, $i$-ta od tih riječi odgovara riječi na kamenoj ploči divljaka s oznakom $i$.

After that, the integer $Q$ ($1 \leqslant Q \leqslant 10^5$) follows, the number of rounds in the game.
%Nakon toga slijedi prirodan broj $Q$ ($1 \leqslant Q \leqslant 10^5$), broj rundi igre.

The following $Q$ lines describe the round of the game, the $i$\textsuperscript{th} line describing the $i$\textsuperscript{th} round of the game.
%Idućih $Q$ redaka opisuju runde igre, $i$-ti od tih redaka opisuje $i$-tu rundu igre.
Each line will contain the integer $O$. In the case when $O$ is equal to 1, it denotes the first type of round and the shown word $P$ follows in the same line, consisting of only lowercase letters of the English alphabet.
%U svakom retku nalazit će se broj $O$. U slučaju da je $O$ jednak 1 radi se o prvom tipu runde i u istom retku slijedi pokazana riječ $P$ koja se sastoji samo od malih slova engleske abecede.

In the case when $O$ is equal to 2, it denotes the second type of round and the number $S$ ($1 \leqslant S \leqslant N$) follows in the same line, the label of the barbarian whom Tarzan asked the question.
%U slučaju da je $O$ jednak 2 radi se o drugom tipu runde i u istom retku slijedi broj $S$ ($1 \leqslant S \leqslant N$), oznaka divljaka kojemu je Tarzan postavio pitanje.

The total length of all words written on the barbarians' stone tablets will not exceed $2\cdot10^6$.\\
%Ukupna duljina svih riječi koje pišu na pločama divljaka neće biti veća od $2\cdot10^6$.
The total length of all words that Tarzan shows to the barbarians will not exceed $2\cdot10^6$.
%Ukupna duljina svih riječi koje Tarzan pokazuje divljacima neće biti veća od $2\cdot10^6$.

\strut

\naslov{output}

For each round of a different form, output a single line. The $i$\textsuperscript{th} line must contain the correct answer to Tarzan's question in the $i$\textsuperscript{th} round of type 2.
%Za svaku rundu drugog oblika ispišite po jedan redak. U $i$-tom ispisanom retku mora se nalaziti točan odgovor na Tarzanovo pitanje u $i$-toj rundi drugog oblika.

\strut

\naslov{scoring}

In test cases worth 50\% of total points, it will hold $N \leqslant 20\,000$.
%U test podacima ukupno vrijednima 50\% bodova vrijedit će $N \leqslant 20\,000$.

\pagebreak

\naslov{sample tests}

\begin{center}
\fontfamily{\ttdefault}
\fontsize{10pt}{1em}
\selectfont
\begin{tabu}to 0.99\textwidth{|X[1]|X[1]|}
\hline
& \\ 
\rowfont{\fontsize{10pt}{1em}\bfseries}
input & input \\
\verbatiminput{divljak/divljak.dummy.in.1} & 
\verbatiminput{divljak/divljak.dummy.in.2} \\
\rowfont{\fontsize{10pt}{1em}\bfseries}
output & output \\
\verbatiminput{divljak/divljak.dummy.out.1} & 
\verbatiminput{divljak/divljak.dummy.out.2} \\
\hline
\end{tabu}
\end{center}

{
\fontsize{10pt}{1em}
\selectfont
\textbf{Clarification of the first example:} The only word Tarzan has spoken is abca. The answer to the first question is, of course, 1 because the word a is a substring of the word abca. The answer to the second question is also 1 because the word abc is a substring of the word abca. \\
%{Pojašnjenje prvog primjera:} Jedina riječ koju je Tarzan izrekao je abca. Odgovor na prvi upit je naravno 1 jer se riječ a nalazi u riječi abca. Odgovor na drugi upit je također 1 jer se riječ abc nalazi u riječi abca. \\
%\textbf{Pojašnjenje drugog primjera:} pojašnjenje pojašnjenje pojašnjenje
}