\renewcommand{\taskname}{UŽAS}
\renewcommand{\timelimit}{1 sekunda}
\renewcommand{\memorylimit}{32 MB}
\renewcommand{\score}{30 bodova}

Živite u Ulici brijestova na kućnom broju $X$ i želite doći do svojega prijatelja koji živi na kućnom broju $Y$ na istroj strani ulice. Na putu nećete nikada prelaziti cestu, jer ste već na pravoj strani.

Opće je poznato da su kuće čiji su kućni brojevi djeljivi s 4 uklete. Izračunajte kraj koliko ukletih kuća ćete proći na putu od svoje kuće do prijatelja (računajući i vašu i prijateljevu kuću ako su uklete). 

\textbf{Napomena:} s jedne strane ulice nalaze se parni kućni brojevi, a s druge strane neparni.

\strut

\naslov{ulazni podaci}

U jedinom retku ulaza nalaze se dva različita prirodna broja $X$ i $Y$ ($1 \leqslant X, Y \leqslant 1\,000$), vaš i prijateljev kućni broj.

\naslov{izlazni podaci}

U prvi i jedini redak izlaza ispišite jedan cijeli broj, broj ukletih kuća kraj kojih ćete proći na putu.

\strut

\naslov{primjeri test podataka}

\begin{center}
\fontfamily{\ttdefault}
\fontsize{10pt}{1em}
\selectfont
\begin{tabu}to 0.99\textwidth{|X[1]|X[1]|}
\hline
& \\ 
\rowfont{\fontsize{10pt}{1em}\bfseries}
ulaz & ulaz \\
\verbatiminput{uzas/uzas.dummy.in.1} & 
\verbatiminput{uzas/uzas.dummy.in.2} \\
\rowfont{\fontsize{10pt}{1em}\bfseries}
izlaz & izlaz \\
\verbatiminput{uzas/uzas.dummy.out.1} & 
\verbatiminput{uzas/uzas.dummy.out.2} \\
\hline
\end{tabu}
\end{center}

{
\fontsize{10pt}{1em}
\selectfont
\textbf{Pojašnjenje prvog primjera:} Proći ćete kraj ukletih kuća sa kućnim brojevima 4, 6 i 8. \\
%\textbf{Pojašnjenje drugog primjera:} pojašnjenje pojašnjenje pojašnjenje
}