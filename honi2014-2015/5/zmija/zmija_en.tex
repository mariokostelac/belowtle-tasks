\renewcommand{\taskname}{ZMIJA}
\renewcommand{\timelimit}{1 second}
\renewcommand{\memorylimit}{32 MB}
\renewcommand{\score}{80 points}

Mirko is making a copy of the popular computer game "Snake". In the game, you control the movement of a snake on a screen with dimensions of $R \cdot S$ pixels. The objective of the game is to collect all the apples.
%Mirko radi kopiju popularne računalne igre "Zmijica". U igri kontrolirate zmijičine kretnje po ekranu dimenzija $R \cdot S$ piksela, a cilj igre je pokupiti sve jabučice koje se nalaze na ekranu. 

Unfortunately, Mirko's implementation isn't that great and the gameplay is different than the original. Here is a description of Mirko's game:
%Nažalost, Mirkova implementacija nije baš bajna pa se tijek igre razlikuje od originala. Slijedi opis Mirkove igre:
\begin{itemize}
\item unlike the original, the apples don't appear randomly on the screen, but instead you know the positions of all apples at the beginning of the game
%\item za razliku od originala, jabučice se ne pojavljuju nasumično na ekranu, već su na početku igre poznate pozicije svih jabuka
\item at the beginning of the game, the snake is located at the lower left edge of the screen and is facing right
%\item na početku igre, zmijica se nalazi u donjem lijevom kutu ekrana i gleda udesno
\item there are two buttons in the game, denoted with A and B
%\item postoje dvije tipke u igri, označene s A i B
\item pressing the button A, the snake moves forward by 1 pixel in the direction which it is currently facing. If that move would cause the snake to go off screen, nothing happens.
%\item pritiskom na tipku A, zmijica se pomiče za jedan piksel u smjeru u kojem trenutno gleda. Ako bi zmijica tim pomakom izašla iz ekrana, onda se ne događa ništa.
\item pressing the button B, the snake moves up by 1 pixel and changes the direction it's facing by 180\textdegree
%\item pritiskom na tipku B, zmijica se pomiče za jedan piksel prema gore i mijenja smjer gledanja za 180\textdegree
\item when the snake moves to a pixel containing an apple, it eats the apple but doesn't grow like in the original game
%\item kada se zmijica pomakne na piksel na kojem se nalazi jabuka, pojede ju, ali ne naraste kao u originalu
\end{itemize}

You have the following task: for given positions of apples at the beginning of the game, determine \textbf{the smallest number of button presses} it takes for the snake to collect \textbf{all the apples}.
%Vaš zadatak je sljedeći: za zadane pozicije jabučica na početku igre odredite \textbf{najmanji broj pritisaka tipki} potreban da zmijica skupi \textbf{sve jabučice}.

\strut

\naslov{input}

The first line of input contains the integers $R$ and $S$ ($2 \leqslant R, S \leqslant 1\,000$), the width and height of the screen.
%U prvom retku nalaze se prirodni brojevi $R$ i $S$ ($2 \leqslant R, S \leqslant 1\,000$), visina i širina ekrana.

Each of the following $R$ lines contains exactly $S$ characters. These characters represent the content of the screen. Pixels with apples on them are denoted with 'J' and empty pixels are denoted with '.'.
%U svakom od idućih $R$ redaka nalazi se točno $S$ znakova. Ti znakovi predstavljaju sadržaj ekrana. Na pikselima gdje su jabučice bit će znak 'J', na praznim pikselima znak '.'.

The lower left corner contains the character 'Z' that represents the snake in its initial position.
%U donjem lijevom kutu nalazit će se znak 'Z' koji predstavlja zmijicu na njenoj početnoj poziciji.

\strut

\naslov{output}

The first and only line of output must contain the required minimal number of button presses.
%U jedini redak outputa ispišite traženi najmanji broj pritisaka tipki.

\strut

\naslov{sample tests}

\begin{center}
\fontfamily{\ttdefault}
\fontsize{10pt}{1em}
\selectfont
\begin{tabu}to 0.99\textwidth{|X[1]|X[1]|X[1]|}
\hline
& & \\ 
\rowfont{\fontsize{10pt}{1em}\bfseries}
input & input & input\\
\verbatiminput{zmija/zmija.dummy.in.1} &
\verbatiminput{zmija/zmija.dummy.in.2} & 
\verbatiminput{zmija/zmija.dummy.in.3} \\
\rowfont{\fontsize{10pt}{1em}\bfseries}
output & output & output\\
\verbatiminput{zmija/zmija.dummy.out.1} &
\verbatiminput{zmija/zmija.dummy.out.2} & 
\verbatiminput{zmija/zmija.dummy.out.3} \\
\hline
\end{tabu}
\end{center}

{
\fontsize{10pt}{1em}
\selectfont
\textbf{Clarification of the first example:} The shortest sequence of button presses needed for the snake to collect all the apples is BBAAABB. \\
%{Pojašnjenje prvog primjera:} Najkraći niz pritisaka tipki potreban da zmijica skupi sve jabučice je BBAAABB. \\
%\textbf{Pojašnjenje drugog primjera:} pojašnjenje pojašnjenje pojašnjenje
}